\documentclass[10pt, a4paper]{article}
\usepackage{amsmath}
\usepackage{amssymb}
\usepackage{amsfonts}
\usepackage{amsthm}
\usepackage{mathrsfs}
\usepackage{mathtools}
\usepackage{bbm}
\usepackage{listings}
\usepackage{enumitem} % reference enum item
\usepackage{wasysym} % smiley faces!

% COLORS
\usepackage{color}
\usepackage{xcolor}
\definecolor{mygray}{rgb}{0.4,0.4,0.4}

% Math Utils
\newcommand{\inner}[2]{\langle#1,\,#2\rangle}
\newcommand{\innerone}[1]{\langle#1\rangle}
\newcommand{\paren}[1]{\left(#1\right)}
\newcommand{\bracket}[1]{\left[#1\right]}
\newcommand{\given}{\biggr\rvert}
\newcommand{\PP}[1]{\mathbb{P}\paren{#1}}
\newcommand{\PPB}[1]{\mathbb{P}\bracket{#1}}
\newcommand{\PPP}{\mathbb{P}}
\newcommand{\E}{\mathbb{E}}
\newcommand{\EE}[1]{\mathbb{E}\paren{#1}}
\newcommand{\EEB}[1]{\mathbb{E}\bracket{#1}}
\newcommand{\Tr}[1]{\text{Tr}\paren{#1}}
\newcommand{\R}{\mathbb{R}}
\newcommand{\Z}{\mathbb{Z}}
\newcommand{\Q}{\mathbb{Q}}
\newcommand{\N}{\mathbb{N}}
\newcommand{\xeq}[1]{\stackrel{#1}{=}}
\newcommand{\indist}{\mathcal{D}}
\newcommand{\inprob}{\mathcal{P}}
\newcommand{\borel}{\mathcal{B}}
\newcommand{\indic}[1]{\mathbbm{1}_{#1}}
\newcommand\restr[2]{{% we make the whole thing an ordinary symbol
 \left.\kern-\nulldelimiterspace% automatically resize the bar with \right
 #1 % the function
 \vphantom{\big|} % pretend it's a little taller at normal size
 \right|_{#2} % this is the delimiter
 }}
\newcommand{\im}{\text{Im}}
\newcommand{\cov}[2]{\text{Cov}\paren{#1,#2}}
\newcommand{\corr}[2]{\text{Corr}\paren{#1,#2}}
\newcommand{\var}[1]{\text{Var}\paren{#1}}
\newcommand{\abs}[1]{\left\lvert#1\right\rvert}
\newcommand{\norm}[1]{\vert\vert#1\rvert\rvert}
\newcommand{\grad}{\nabla}
\newcommand{\varrow}{\overset{\rightharpoonup}}
\newcommand{\diag}[1]{\text{diag}\paren{#1}}
\newcommand{\mesh}{\text{mesh}}
\newcommand{\sgn}[1]{\text{sgn}\paren{#1}}
\newcommand{\set}[1]{\{#1\}}
\newcommand{\swap}{\operatorname{swap}}

\DeclareMathOperator*{\argmax}{arg\,max}
\DeclareMathOperator*{\argmin}{arg\,min}
\DeclareMathOperator*{\tv}{d_{TV}}

\DeclarePairedDelimiter{\floor}{\lfloor}{\rfloor}
\DeclarePairedDelimiter{\ceil}{\lceil}{\rceil}

\newcommand\indep{\protect\mathpalette{\protect\independenT}{\perp}}
\def\independenT#1#2{\mathrel{\rlap{$#1#2$}\mkern2mu{#1#2}}}

\newcommand{\code}[1]{{\texttt{\textcolor{mygray}{\small #1}}}}

% TODO highlighting 
\newcommand{\todo}[1]{\textcolor{red}{#1}}

% Theorem stuff
\newtheorem{theorem}{Theorem}[section]
\newtheorem{definition}[theorem]{Definition}
\newtheorem{corollary}{Corollary}[theorem]
\newtheorem{lemma}[theorem]{Lemma}
\newtheorem{exercise}{Exercise}[section]
\newtheorem{remark}{Remark}[section]
\newtheorem{example}{Example}[section]

\usepackage{fullpage}

\begin{document}
\title{Exponential Control k Treatment Model}
\author{James Yang}
\maketitle

\section*{Grid Space}

We will take the grid-space to be $(\xi_1, \xi_2)$
where $\xi_1 = \log(\lambda_c)$ ($\lambda_c$ is the control arm's $\lambda$),
and $\xi_2 = \log(h)$ ($h = \lambda_t / \lambda_c$ is the hazard rate,
i.e.\ the ratio of treatment to control $\lambda$ parameter).
Note that the grid-space is not in the natural parameter space.
This results in a slightly different upper-bound formulation,
which is outlined below.

\section*{Upper Bound}

The 0th order Monte Carlo term and its upper bound
need no change from this reparametrization.

\subsection*{Gradient Term}

Let $\xi = (\xi_1, \xi_2)$, $\lambda = (\lambda_c, \lambda_t)$,
and $A$ be the event of false rejection.
\begin{align*}
    \nabla_{\xi} P_\lambda(A) 
    &=
    \nabla_{\xi} \int_A \frac{P_{\lambda}}{P_{\lambda_0}} dP_{\lambda_0}
    =
    \int_A \nabla_{\xi} \frac{P_{\lambda}}{P_{\lambda_0}} dP_{\lambda_0}
    \\&=
    \int_A (D_{\xi}\lambda)^\top \nabla_{\lambda} \frac{P_{\lambda}}{P_{\lambda_0}} dP_{\lambda_0}
\end{align*}
If $\xi_0$ is the point at which we are Taylor expanding,
it suffices to compute this gradient at $\xi = \xi_0$ (equivalently, $\lambda_0$).
This results in

\begin{align*}
    \nabla_{\xi} P_{\lambda_0}(A) 
    &=
    \int_A (D_{\xi}\lambda(\xi_0))^\top (T - \nabla_\lambda A(\lambda_0)) dP_{\lambda_0}
\end{align*}

Hence, our gradient Monte Carlo estimate will be
\begin{align*}
    \hat{\nabla f}
    :=
    \frac{1}{N}
    \sum\limits_{i=1}^N
    D_{\xi} \lambda(\xi_0)^\top (T(X_i)-\nabla_\lambda A(\lambda_0)) \indic{X_i \in A}
\end{align*}

\subsection*{Hessian Term}


\end{document}
